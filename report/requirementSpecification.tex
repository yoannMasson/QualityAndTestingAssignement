%Copyright 2014 Jean-Philippe Eisenbarth
%This program is free software: you can 
%redistribute it and/or modify it under the terms of the GNU General Public 
%License as published by the Free Software Foundation, either version 3 of the 
%License, or (at your option) any later version.
%This program is distributed in the hope that it will be useful,but WITHOUT ANY 
%WARRANTY; without even the implied warranty of MERCHANTABILITY or FITNESS FOR A 
%PARTICULAR PURPOSE. See the GNU General Public License for more details.
%You should have received a copy of the GNU General Public License along with 
%this program.  If not, see <http://www.gnu.org/licenses/>.

%Based on the code of Yiannis Lazarides
%http://tex.stackexchange.com/questions/42602/software-requirements-specification-with-latex
%http://tex.stackexchange.com/users/963/yiannis-lazarides
%Also based on the template of Karl E. Wiegers
%http://www.se.rit.edu/~emad/teaching/slides/srs_template_sep14.pdf
%http://karlwiegers.com
\documentclass{scrreprt}
\usepackage{listings}
\usepackage{booktabs}
\usepackage{underscore}
\usepackage[bookmarks=true]{hyperref}
\usepackage[utf8]{inputenc}
\usepackage[english]{babel}
\hypersetup{
    bookmarks=false,    % show bookmarks bar?
    pdftitle={Software Requirement Specification},    % title
    pdfauthor={Yoann Masson},                     % author
    pdfsubject={Scheduler Simulation for IT departement},% subject of the document
    pdfkeywords={}, % list of keywords
    colorlinks=true,       % false: boxed links; true: colored links
    linkcolor=blue,       % color of internal links
    citecolor=black,       % color of links to bibliography
    filecolor=black,        % color of file links
    urlcolor=purple,        % color of external links
    linktoc=page            % only page is linked
}%
\def\myversion{1.0 }
\date{}
%\title
\usepackage{hyperref}
\begin{document}

\begin{flushright}
    \rule{16cm}{5pt}\vskip1cm
    \begin{bfseries}
        \Huge{SOFTWARE REQUIREMENTS\\ SPECIFICATION}\\
        \vspace{1.9cm}
        for\\
        \vspace{1.9cm}
        Scheduler simulation for the Cranfield IT Departement\\
        \vspace{1.9cm}
        \LARGE{Version \myversion approved}\\
        \vspace{1.9cm}
        Prepared by Yoann Masson\\
        \today\\
    \end{bfseries}
\end{flushright}

\tableofcontents

\chapter{Introduction}

Nowadays high powerfull computer is highly interesting for industries and research. To stay on top of the game, Cranfield University is purchasing a new super computer. It will be used by a large number of users ( students, researchers, etc...), so a good ressources management and a good billing policy is mandatory. In order to achieve these goals, a simulation of the new job control system is requiered. 

This document will stated the requirement specifications of the simulation of the new job control system. The software shall allow IT supports to run different simulations of the job scheduler for the new supercomputer and shall keep track of the accounting state at any point in time. The simulation shall model the behaviour of the computing platform to explore alternative accounting strategies.
The requirements should be well explained and not subject to any interpretation.



\chapter{Overall Description}

\section{Product Perspective}
$<$Describe the context and origin of the product being specified in this SRS.  
For example, state whether this product is a follow-on member of a product 
family, a replacement for certain existing systems, or a new, self-contained 
product. If the SRS defines a component of a larger system, relate the 
requirements of the larger system to the functionality of this software and 
identify interfaces between the two. A simple diagram that shows the major 
components of the overall system, subsystem interconnections, and external 
interfaces can be helpful.$>$

\section{Product Functions}
$<$Summarize the major functions the product must perform or must let the user 
perform. Details will be provided in Section 3, so only a high level summary 
(such as a bullet list) is needed here. Organize the functions to make them 
understandable to any reader of the SRS. A picture of the major groups of 
related requirements and how they relate, such as a top level data flow diagram 
or object class diagram, is often effective.$>$

\section{User Classes and Characteristics}
$<$Identify the various user classes that you anticipate will use this product.  
User classes may be differentiated based on frequency of use, subset of product 
functions used, technical expertise, security or privilege levels, educational 
level, or experience. Describe the pertinent characteristics of each user class.  
Certain requirements may pertain only to certain user classes. Distinguish the 
most important user classes for this product from those who are less important 
to satisfy.$>$

\section{Operating Environment}
$<$Describe the environment in which the software will operate, including the 
hardware platform, operating system and versions, and any other software 
components or applications with which it must peacefully coexist.$>$

\section{Design and Implementation Constraints}
$<$Describe any items or issues that will limit the options available to the 
developers. These might include: corporate or regulatory policies; hardware 
limitations (timing requirements, memory requirements); interfaces to other 
applications; specific technologies, tools, and databases to be used; parallel 
operations; language requirements; communications protocols; security 
considerations; design conventions or programming standards (for example, if the 
customer’s organization will be responsible for maintaining the delivered 
software).$>$

\section{User Documentation}
$<$List the user documentation components (such as user manuals, on-line help, 
and tutorials) that will be delivered along with the software. Identify any 
known user documentation delivery formats or standards.$>$
\section{Assumptions and Dependencies}

$<$List any assumed factors (as opposed to known facts) that could affect the 
requirements stated in the SRS. These could include third-party or commercial 
components that you plan to use, issues around the development or operating 
environment, or constraints. The project could be affected if these assumptions 
are incorrect, are not shared, or change. Also identify any dependencies the 
project has on external factors, such as software components that you intend to 
reuse from another project, unless they are already documented elsewhere (for 
example, in the vision and scope document or the project plan).$>$


\chapter{External Interface Requirements}

\section{User Interfaces}
The simulation will run on parameters set by users. All settings will be stored and can be changed in a provided file. When executing the simulation this file must be given as a parameter of the executable for the settings to be taken in account. Parameters will go from the number user to the accounting settings and the number/frequency of jobs.\\
If no file is provided, default settings will be applied. Those default settings can be found as an appendix of this document.\\
The output of the simulation shall be printed on the console executing the simulation.


\chapter{System Features}

\section{User Management}


\subsection{Description}
Simulated users are the heart of the system since they are the one making job requests. They are dividing into three groups :
\begin{itemize}
\item students
\item researchers
\item IT Supports
\end{itemize}
Each user is entitled budget, IT Supports get unlimited budget while students have a budget depending on their Msc. Every researcher has financial ressources according to their groups, some of them can have additional ressources.\\
Student will be separate evenly between these 3 Msc: Computer, Water, Soil.

\subsection{Settings from the input files}
The settings that can be changed through the settings file are:

\begin{table}[ht]
\centering
\caption{User variable}
\begin {tabular}{ l l l}
\toprule
Variable & Default Value & Description \\
\midrule
\midrule
 STUDENT_NUMBER & 60  & the number of student considered in the simulation \\
 RESEARCHER_NUMBER & 20 & the number of researcher considered in the simulation \\
 COMPUTER_BUDGET & 20.00  & Budget attribut to one student from Computer Msc \\
 WATER_BUDGET & 10.00  & Budget attribut to one student from Water Msc \\
 SOIL_BUDGET & 5.00  & Budget attribut to one student from Soil Msc \\
\bottomrule
\end {tabular}
\end{table}


\subsection{Functional Requirements}
Every variable must be greater or equal than 0


\section{Ressource Management}


\subsection{Description}
With this feature, the user can manage the power of the simulated computing system. The changes that can be made are about the numbers of nodes, the price of using these nodes and the cost for the IT departement for running the super computer. 
\subsection{Settings from the input files}
The settings that can be changed through the settings file are:

\begin{table}[ht]
\centering
\caption{Computing variable}
\begin {tabular}{ l l l}
\toprule
Variable & Default Value & Description \\
\midrule
\midrule
NODES_16_PROCESSOR_NUMBER & 80  & the number of traditional nodes\\&& with 16 processors considered \\
NODES_32_PROCESSOR_NUMBER & 40  & the number of traditional nodes\\&& with 32 processors considered  \\
NODES_64_PROCESSOR_NUMBER & 30  & the number of traditional nodes\\&& with 64 processors considered \\
ACCELERATED_16_PROCESSOR_NUMBER & 30  & the number of accelerated\\&& nodes with 16 processors considered  \\
ACCELERATED_32_PROCESSOR_NUMBER & 20  & the number of accelerated\\&& nodes with 32 processors considered  \\
ACCELERATED_64_PROCESSOR_NUMBER & 10  & the number of accelerated\\&& nodes with 64 processors considered \\
SPECIALIZED_NUMBER & 20  & the number of specialized\\&& nodes considered \\
SMALL_PRICE & 0.05  & the price of one machine hour\\&& for computing small jobs. \\
MEDIUM_PRICE & 0.05  & the price of one machine hour\\&& for computing medium jobs. \\
LARGE_PRICE & 0.05 & the price of one machine hour\\&& for computing large jobs. \\
HUGE_PRICE & 0.05  & the price of one machine hour\\&& for computing huge jobs. \\
COMPUTER_COST & 0.01  & the cost for the IT Department\\&& per node\&hour. \\

\bottomrule
\end {tabular}
\end{table}


\subsection{Functional Requirements}
Every variable must be greater or equal than 0. There must be at least 128 nodes in total.

\section{Job genreation Management}


\subsection{Description}
Jobs are to be generated randomly, the more users you have the more jobs are generated. Jobs are generated with an exponential distribution with parameters dependings on the user settings. 
\subsection{Settings from the input files}
The settings that can be changed through the settings file are:

\begin{table}[ht]
\centering
\caption{Computing variable}
\begin {tabular}{ l l l}
\toprule
Variable & Default Value & Description \\
\midrule
\midrule
NODES_16_PROCESSOR_NUMBER & 80  & the number of traditional nodes\\&& with 16 processors considered \\
NODES_32_PROCESSOR_NUMBER & 40  & the number of traditional nodes\\&& with 32 processors considered  \\
NODES_64_PROCESSOR_NUMBER & 30  & the number of traditional nodes\\&& with 64 processors considered \\
ACCELERATED_16_PROCESSOR_NUMBER & 30  & the number of accelerated\\&& nodes with 16 processors considered  \\
ACCELERATED_32_PROCESSOR_NUMBER & 20  & the number of accelerated\\&& nodes with 32 processors considered  \\
ACCELERATED_64_PROCESSOR_NUMBER & 10  & the number of accelerated\\&& nodes with 64 processors considered \\
SPECIALIZED_NUMBER & 20  & the number of specialized\\&& nodes considered \\
SMALL_PRICE & 0.05  & the price of one machine hour\\&& for computing small jobs. \\
MEDIUM_PRICE & 0.05  & the price of one machine hour\\&& for computing medium jobs. \\
LARGE_PRICE & 0.05 & the price of one machine hour\\&& for computing large jobs. \\
HUGE_PRICE & 0.05  & the price of one machine hour\\&& for computing huge jobs. \\
COMPUTER_COST & 0.01  & the cost for the IT Department\\&& per node\&hour. \\

\bottomrule
\end {tabular}
\end{table}


\subsection{Functional Requirements}
Every variable must be greater or equal than 0. There must be at least 128 nodes in total.

\chapter{Other Nonfunctional Requirements}

\section{Performance Requirements}
$<$If there are performance requirements for the product under various 
circumstances, state them here and explain their rationale, to help the 
developers understand the intent and make suitable design choices. Specify the 
timing relationships for real time systems. Make such requirements as specific 
as possible. You may need to state performance requirements for individual 
functional requirements or features.$>$

\section{Safety Requirements}
$<$Specify those requirements that are concerned with possible loss, damage, or 
harm that could result from the use of the product. Define any safeguards or 
actions that must be taken, as well as actions that must be prevented. Refer to 
any external policies or regulations that state safety issues that affect the 
product’s design or use. Define any safety certifications that must be 
satisfied.$>$

\section{Security Requirements}
$<$Specify any requirements regarding security or privacy issues surrounding use 
of the product or protection of the data used or created by the product. Define 
any user identity authentication requirements. Refer to any external policies or 
regulations containing security issues that affect the product. Define any 
security or privacy certifications that must be satisfied.$>$

\section{Software Quality Attributes}
$<$Specify any additional quality characteristics for the product that will be 
important to either the customers or the developers. Some to consider are: 
adaptability, availability, correctness, flexibility, interoperability, 
maintainability, portability, reliability, reusability, robustness, testability, 
and usability. Write these to be specific, quantitative, and verifiable when 
possible. At the least, clarify the relative preferences for various attributes, 
such as ease of use over ease of learning.$>$

\section{Business Rules}
$<$List any operating principles about the product, such as which individuals or 
roles can perform which functions under specific circumstances. These are not 
functional requirements in themselves, but they may imply certain functional 
requirements to enforce the rules.$>$


\chapter{Other Requirements}
$<$Define any other requirements not covered elsewhere in the SRS. This might 
include database requirements, internationalization requirements, legal 
requirements, reuse objectives for the project, and so on. Add any new sections 
that are pertinent to the project.$>$

\section{Appendix A: Glossary}
%see https://en.wikibooks.org/wiki/LaTeX/Glossary
$<$Define all the terms necessary to properly interpret the SRS, including 
acronyms and abbreviations. You may wish to build a separate glossary that spans 
multiple projects or the entire organization, and just include terms specific to 
a single project in each SRS.$>$

\section{Appendix B: Analysis Models}
$<$Optionally, include any pertinent analysis models, such as data flow 
diagrams, class diagrams, state-transition diagrams, or entity-relationship 
diagrams.$>$

\section{Appendix C: To Be Determined List}
$<$Collect a numbered list of the TBD (to be determined) references that remain 
in the SRS so they can be tracked to closure.$>$

\end{document}
