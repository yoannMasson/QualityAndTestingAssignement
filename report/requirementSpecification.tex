%Copyright 2014 Jean-Philippe Eisenbarth
%This program is free software: you can 
%redistribute it and/or modify it under the terms of the GNU General Public 
%License as published by the Free Software Foundation, either version 3 of the 
%License, or (at your option) any later version.
%This program is distributed in the hope that it will be useful,but WITHOUT ANY 
%WARRANTY; without even the implied warranty of MERCHANTABILITY or FITNESS FOR A 
%PARTICULAR PURPOSE. See the GNU General Public License for more details.
%You should have received a copy of the GNU General Public License along with 
%this program.  If not, see <http://www.gnu.org/licenses/>.

%Based on the code of Yiannis Lazarides
%http://tex.stackexchange.com/questions/42602/software-requirements-specification-with-latex
%http://tex.stackexchange.com/users/963/yiannis-lazarides
%Also based on the template of Karl E. Wiegers
%http://www.se.rit.edu/~emad/teaching/slides/srs_template_sep14.pdf
%http://karlwiegers.com
\documentclass{scrreprt}
\usepackage{listings}
\usepackage{booktabs}
\usepackage{underscore}
\usepackage[bookmarks=true]{hyperref}
\usepackage[utf8]{inputenc}
\usepackage[english]{babel}
\hypersetup{
    bookmarks=false,    % show bookmarks bar?
    pdftitle={Software Requirement Specification},    % title
    pdfauthor={Yoann Masson},                     % author
    pdfsubject={Scheduler Simulation for IT departement},% subject of the document
    pdfkeywords={}, % list of keywords
    colorlinks=true,       % false: boxed links; true: colored links
    linkcolor=blue,       % color of internal links
    citecolor=black,       % color of links to bibliography
    filecolor=black,        % color of file links
    urlcolor=purple,        % color of external links
    linktoc=page            % only page is linked
}%
\def\myversion{1.0 }
\date{}
%\title
\usepackage{hyperref}
\begin{document}

\begin{flushright}
    \rule{16cm}{5pt}\vskip1cm
    \begin{bfseries}
        \Huge{SOFTWARE REQUIREMENTS\\ SPECIFICATION}\\
        \vspace{1.9cm}
        for\\
        \vspace{1.9cm}
        Scheduler simulation for the Cranfield IT Departement\\
        \vspace{1.9cm}
        \LARGE{Version \myversion approved}\\
        \vspace{1.9cm}
        Prepared by Yoann Masson\\
        \today\\
    \end{bfseries}
\end{flushright}

\tableofcontents

\chapter{Introduction}

Nowadays high powerful computer is highly interesting for industries and research. To stay on top of the game, Cranfield University is purchasing a new super computer. It will be used by a large number of users ( students, researchers, etc...), so a good resources management and a good billing policy is mandatory. In order to achieve these goals, a simulation of the new job control system is required. 

This document will stated the requirement specifications of the simulation of the new job control system. The software shall allow IT supports to run different simulations of the job scheduler for the new supercomputer and shall keep track of the accounting state at any point in time. The simulation shall model the behaviour of the computing platform to explore alternative accounting strategies.
The requirements should be well explained and not subject to any interpretation.



\chapter{Overall Description}

\section{Product Functions}

The simulation will randomly create jobs based on an exponential distribution law, meaning that the resources needed and the time needed to run the job will depends on parameters given by the real user of the simulation. The cost of running a job is up to these two parameters.\\
As in real life, the job system control simulation will have various type of user with each having a different budget, users that can't pay for a job running will not be able to see their job running as it would be on real life.\\
All the ressources shall be split between three queue:\begin{itemize}
\item small queue, 20\% of all cores working on small jobs ( less than 2 nodes and 1 hour )
\item medium queue, 30\% of all cores working for medium jobs ( less than 10\% of all cores and less than 8h)
\item large queue, 50\% of all cores working on large jobs (less than 50\% of all cores and less than 16h)
\end{itemize}
Every other jobs will run on the week-end (from friday 17:00 to monday 9:00 ), where all the queue are not working to let the huge queue works with huge jobs that can take up to 48h and all resources.\\
The queuing system shall work on the first come/first serve strategy, the firsts job to request resources will be given the resources until there is not enough resources for the next job that will needs to wait for one to finish.

\section{Interaction between classes}
Classes work together, here is a data flow model to briefly see the interactions between them.\\INSERT DATA FLOW

\section{Operating Environment}
The simulation shall run under Java 1.7 in a terminal or in Eclipse by importing the project. For the simulation to take in account the settings file you must provide it as an argument when lauching the programm.


\section{User Documentation}
The software shall be delivered with the document of the class composing the project. This documentation shall be generate with javadoc, providing comments about methods, argument, return statement and exceptions


\chapter{External Interface Requirements}

\section{User Interfaces}
The simulation will run on parameters set by users. All settings will be stored and can be changed in a provided file. When executing the simulation this file must be given as a parameter of the executable for the settings to be taken in account. Parameters will go from the number user to the accounting settings and the number/frequency of jobs.\\
If no file is provided, default settings will be applied. Those default settings can be found later in this document.\\
The output of the simulation shall be printed on the console executing the simulation.


\chapter{System Features}

\section{User Management}


\subsection{Description}
Simulated users are the heart of the system since they are the one making job requests. They are dividing into two groups :
\begin{itemize}
\item students
\item researchers
\end{itemize}
Each user is entitled budget, students have a budget depending on their Msc. Every researcher has financial resources according to their groups, some of them can have additional resources.\\
Student will be separate evenly between these 3 Msc: Computer, Water, Soil.

\subsection{Settings from the input files}
The settings that can be changed through the settings file are:

\begin{table}[ht]
\centering
\caption{User variable}
\begin {tabular}{ l l l}
\toprule
Variable & Default Value & Description \\
\midrule
\midrule
 STUDENT_NUMBER & 60  & the number of student considered in the simulation \\
 RESEARCHER_NUMBER & 20 & the number of researcher considered in the simulation \\
 COMPUTER_BUDGET & 20.00  & Budget attribute to one student from Computer Msc \\
 WATER_BUDGET & 10.00  & Budget attribute to one student from Water Msc \\
 SOIL_BUDGET & 5.00  & Budget attribute to one student from Soil Msc \\
 RESEARCHER_BUDGET & 50.00  & Budget attribute to one researcher \\
\bottomrule
\end {tabular}
\end{table}
Every variable must be greater or equal than 0

\subsection{Functional Requirements}
Here is the list of functional requirement that the class user must provide:\begin{itemize}
    \item A user must be able to tell at any moment his budget
    \item A user must be able to give his status (Msc, researcher) at any moment
    \item A user must be able to pay for a job
\end{itemize}
\subsection{Non-Functional Requirements}
If a user can not pay for a job, he is not charged for it. He must notice the caller of the pay method with an exception stating that the user does not have enough budget

\section{Resource Management}


\subsection{Description}
With this feature, the user can manage the power of the simulated computing system. The changes that can be made are about the numbers of nodes, the price of using these nodes and the cost for the IT department for running the super computer. 
\subsection{Settings from the input files}
The settings that can be changed through the settings file are:

\begin{table}[ht]
\centering
\caption{Computing variable}
\begin {tabular}{ l l l}
\toprule
Variable & Default Value & Description \\
\midrule
\midrule
NODES_PROCESSOR_NUMBER & 80  & the number of traditional nodes\\
ACCELERATED_PROCESSOR_NUMBER & 40  & the number of accelerated nodes\\ 
SPECIALIZED_NUMBER & 20  & the number of specialized nodes considered \\
SMALL_PRICE & 0.05  & the price of one machine hour\\&& for computing small jobs. \\
MEDIUM_PRICE & 0.05  & the price of one machine hour\\&& for computing medium jobs. \\
LARGE_PRICE & 0.05 & the price of one machine hour\\&& for computing large jobs. \\
HUGE_PRICE & 0.05  & the price of one machine hour\\&& for computing huge jobs. \\
COMPUTER_COST & 0.01  & the cost for the IT Department\\&& per node\&hour. \\

\bottomrule
\end {tabular}
\end{table}
Every variable must be greater or equal than 0. There must be at least 128 nodes in total.

\subsection{Functional Requirements}


\section{Simulation Management}

\subsection{Description}
In this section, we will see the parameters that can be set to manage the simulation. First we need to manage the job generation. Jobs are generated with an exponential distribution with parameters depending on the user settings. We can change the duration of the simulation, the more time the more jobs are submitted throughout the entire simulation.
\subsection{Settings from the input files}
The settings that can be changed through the settings file are:

\begin{table}[ht]
\centering
\caption{Computing variable}
\begin {tabular}{ l l l }
\toprule
Variable & Default Value & Description \\
\midrule
\midrule
ENTIRE_TIME & 8 & the number of weeks the simulation will last \\
DELTA_SIZE & 1 & the value of delta used in the exponential distribution \\&& to generate the size of a job\\
DELTA_JOB_TIME & 1 & the value of delta used in the exponential distribution\\&& to generate the time needed to run a job\\
DELTA_REQUEST_TIME & 1 & the value of delta used in the exponential \\&& distribution to generate the time between \\&& two consecutive jobs\\ 

\bottomrule
\end {tabular}
\end{table}


\subsection{Functional Requirements}
In any case, jobs shall last between 1 minute and 48 hours. The time between two consecutive job requests shall not exceed 4 hours. Jobs shall not asked for more than the number of nodes in the queue.

\chapter{Other Nonfunctional Requirements}

\section{Performance Requirements}
$<$If there are performance requirements for the product under various 
circumstances, state them here and explain their rationale, to help the 
developers understand the intent and make suitable design choices. Specify the 
timing relationships for real time systems. Make such requirements as specific 
as possible. You may need to state performance requirements for individual 
functional requirements or features.$>$

\section{Safety Requirements}
$<$Specify those requirements that are concerned with possible loss, damage, or 
harm that could result from the use of the product. Define any safeguards or 
actions that must be taken, as well as actions that must be prevented. Refer to 
any external policies or regulations that state safety issues that affect the 
product’s design or use. Define any safety certifications that must be 
satisfied.$>$

\section{Security Requirements}
$<$Specify any requirements regarding security or privacy issues surrounding use 
of the product or protection of the data used or created by the product. Define 
any user identity authentication requirements. Refer to any external policies or 
regulations containing security issues that affect the product. Define any 
security or privacy certifications that must be satisfied.$>$

\section{Software Quality Attributes}
$<$Specify any additional quality characteristics for the product that will be 
important to either the customers or the developers. Some to consider are: 
adaptability, availability, correctness, flexibility, interoperability, 
maintainability, portability, reliability, reusability, robustness, testability, 
and usability. Write these to be specific, quantitative, and verifiable when 
possible. At the least, clarify the relative preferences for various attributes, 
such as ease of use over ease of learning.$>$

\section{Business Rules}
$<$List any operating principles about the product, such as which individuals or 
roles can perform which functions under specific circumstances. These are not 
functional requirements in themselves, but they may imply certain functional 
requirements to enforce the rules.$>$


\chapter{Other Requirements}
$<$Define any other requirements not covered elsewhere in the SRS. This might 
include database requirements, internationalization requirements, legal 
requirements, reuse objectives for the project, and so on. Add any new sections 
that are pertinent to the project.$>$

\section{Appendix A: Glossary}
%see https://en.wikibooks.org/wiki/LaTeX/Glossary
$<$Define all the terms necessary to properly interpret the SRS, including 
acronyms and abbreviations. You may wish to build a separate glossary that spans 
multiple projects or the entire organization, and just include terms specific to 
a single project in each SRS.$>$

\section{Appendix B: Analysis Models}
$<$Optionally, include any pertinent analysis models, such as data flow 
diagrams, class diagrams, state-transition diagrams, or entity-relationship 
diagrams.$>$

\section{Appendix C: To Be Determined List}
$<$Collect a numbered list of the TBD (to be determined) references that remain 
in the SRS so they can be tracked to closure.$>$

\end{document}
